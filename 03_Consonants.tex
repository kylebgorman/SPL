\chapter{Consonants}

\section{Inventory}

\citet{Allen1978}

\citet{Steriade1995}

\section{Geminate consonants}
\sindex[subject]{Geminates}

% bunch of representational questions here

\section{Degemination}
\sindex[subject]{Geminates!Degemination}

\section{Other cluster simplification}

\citet{Heslin1987}

\section{Devoicing}
\sindex[subject]{Voice!Devoicing}

\citet{Allen1978}

\section{Rhotacism}
\sindex[subject]{Rhotacism|(}

\citet{Saussure1877}

%\citet[213f.]{Baldi2002}

(e.g., \citealt{Albright2005}, \citealt[62]{Foley1965}, \citealt{Gruber2006}, \citealt[134]{Heslin1987}, \citealt[377]{Kenstowicz1996}, \citealt{KiparskyInPress}, \citealt[314]{Klausenburger1976}, \citealt[19]{Matthews1972a}, \citealt{Roberts2012}, \citealt{Touratier1975}, \citealt{Watkins1970}).

(e.g., \citealt[73]{Embick2010}, \citealt{Halle1998a})

\sindex[subject]{Rhotacism}

\citet{Baldi1994}
\citet[41f.]{Safarewicz1932}

his lectures on Latin and Greek phonology \citep{Reichler-Beguelin1980} and in the \emph{Cours de linguistique générale} \citep{CLG}

As \citet[54]{Anderson1985} notes, Saussure appears to regard this as evidence that \textsc{Rhotacism} has been lost by the classical era.

\begin{quote}
Quand on dit : « \emph{s} devient \emph{r} en latin », on fait croire que
la rotacisation est inhérente à la nature de la langue, et l'on reste
embarrassé devant des exceptions telles que \emph{causa}, \emph{rīsus},
etc.\footnote{
    ``Whoever says `\emph{s} became \emph{r} in Latin' implies
that rhotacism is inherent to the language, and remains puzzled by
exceptions such as \emph{causa} [`cause'], \emph{rīsus} [`ridiculed'],
etc.''--KBG}
(Saussure \citeyear{CLG}:202)
\end{quote}
\sindex[subject]{Rhotacism}

\citep{KiparskyInPress} but
\citeauthor{KiparskyInPress} (\citeyear{Kiparsky1968,Kiparsky1973a,Kiparsky1982a,Kiparsky1993}) himself.

Historical (e.g., \citealt[\S180]{Leumann1977}, \citealt[\S173]{Sihler1995}, \citealt[\S119]{Sommer1902})

blocking

\citet[42f.]{Cser2010}, \citet[144]{Gruber2006}, \citet[66]{Ito2003}, and \citet[88]{Roberts2012} all endorse this



non-derived

\citet[90]{Blumenfeld2003}
\citet[149]{Gruber2006}
\citet{Roberts2012}
\citet[260f.]{Touratier1971}


Intervocalic \emph{s} occurs regularly in Greek loanwords, which may take either Greek or Latin inflectional suffixes.
For instance, the nom.sg.~form of `music' may be either \emph{mūsicē}, as in Greek, or Latin-like \emph{mūsica}.
With one exception (\emph{tūs}-\emph{tūris} `incense'),\footnote{
    \citeauthor{KiparskyInPress} claims that \emph{tūs}-\emph{tūris} illustrates the productivity of \textsc{Rhotacism}, but it is just as plausible that the borrowing occurred before the actuation of the rhotacising sound change:

\begin{quote}
The substitution of the letter \emph{r} in the oblique case\ldots{}shows
that θύος could not have found its way into Latin later than the fourth
century B.C. \cite[507]{Thiselton-Dyer1911}
\end{quote}}
however, \textsc{Rhotacism} does not apply to foreign roots following Latin inflectional patterns: \emph{ambrosia} `food of the gods', *\emph{asōtus} `libertine' (acc.sg.~\emph{asōtum}), \emph{basis} `pedestal', \emph{basilica} `public hall', \emph{cerasus} `cherry', \emph{gausapa} `woolen cloth', \emph{lasanum} `cooking utensil', \emph{nausea} `nausea', \emph{pausa} `pause', \emph{philosophus} `philosopher', \emph{poēsis} `poetry', \emph{sarīsa} `lance', \emph{seselis} `seseli'.
Intervocalic \emph{s} is also preserved in loanwords from other languages: Germanic \emph{glaesum} `amber', \emph{bisōntes} `wild oxen'; Celtic \emph{gaesī} `javelins', \emph{omāsum} `tripe'.

\sindex[subject]{Rhotacism|)}

\begin{example}[\textsc{Pre-Liquid Shortening} bled by external sandhi]
\vspace{\baselineskip}
[klaːmoː|r=ad.kai|lum wol|wen.dus pe|r=aj.θe.ra|waː.git]

\gll \emph{clāmōr} \emph{ad} \emph{cael-um}          \emph{volv-e-nd-us}                                \emph{per} \emph{aether-a} \emph{vāg-i-t}
     cry         to        heaven-\textsc{acc.sg.} roll-\textsc{t}-\textsc{fut.pass}-\textsc{nom.sg.} through heaven-\textsc{acc.pl.} wail-\textsc{t}-\textsc{3sg.pres.act.indic.}
\glt `he wails a cry fit to roll up to heaven' (fragments of Ennius)
\glend

\end{example}
