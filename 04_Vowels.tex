\chapter{Vowels and glides}
\label{vowels}

This chapter investigates phenomena involving the vocalic segments, vowels and glides.
The following sections consider the inventory (\S\ref{4:inventory}), the relationship between vowels and glides (\S\ref{4:vowel-glide-alternations}), alternations modifying vowel height (\S\ref{4:height-alternations}, and vowel insertion (\S\ref{4:epenthesis}) and deletion (\S\ref{4:truncation}).

\section{Inventory}
\label{4:inventory}

\subsection{Monophthongs}
\sindex[subject]{Monophthongs}

Latin vowels are traditionally divided into two classes, consisting of monophthongs and diphthongs.
The ten monophthongs represent a traditional five-vowel system, with short and long variants at each of the five positions.
The following examples demonstrate that length is lexically contrastive.
%Chapter \ref{declension} demonstrates that monophthong length also plays a role in declension.

\begin{example}[Contrastive monophthong length]
\begin{tabular}{l c ll c ll}

a. && \emph{malus}  & `bad`             && \emph{mālus}  & `apple tree`  \\
b. && \emph{levis}  & `light in weight' && \emph{lēvis}  & `smooth' \\
c. && \emph{pila}   & `ball'            && \emph{pīla}   & `mortar' \\
d. && \emph{os}     & `bone'            && \emph{ōs}     & `mouth'  \\
e. && \emph{luteus} & `muddy'           && \emph{lūteus} & `yellow'
\end{tabular}
\end{example}

Regarding vowel position, there is a three-way height distinction in the back vowels and a two-way distinction in the front vowels; there is no low front vowel.
The height and backness contrasts can be expressed as a system of features, as in (\ref{4:monophthong-features}).

\begin{example}[Monophthong inventory]
\label{4:monophthong-features}
\begin{tabular}{l cc}
                                          & [$-$\textsc{back}] & [$+$\textsc{back}] \\
\cmidrule{2-3}
\pad{}[$+$\textsc{high}, $-$\textsc{low}] & \emph{i}, \emph{ī} & \emph{u}, \emph{ū} \\
\pad{}[$-$\textsc{high}, $-$\textsc{low}] & \emph{e}, \emph{ē} & \emph{o}, \emph{ō} \\
\pad{}[$-$\textsc{high}, $+$\textsc{low}] &                    & \emph{a}, \emph{ā} \\
\end{tabular}
\end{example}

Some comments by Latin grammarians suggest that short \emph{i} and \emph{u} were just shorter than \emph{ī} and \emph{ū}, but were also somewhat lower
\citep[48f.]{Allen1978}.

The Greek letter $\langle$Υ$\rangle$ was usually romanized (and presumably pronounced as) \emph{u}: e.g.,
???
however, some marginal borrowings from Greek were written with \emph{y}, in which case it may have been read as the high front rounded vowel [y].

%\section{The long vowels}
%k\sindex[subject]{Long vowels}

%\subsection{Long monophthongs}
%\sindex[subject]{Monophthongs}

% FIXME explain how to encode length

\subsection{Diphthongs}
\sindex[subject]{Diphthongs}

productive and those which are not.

Of these, 
all but two---\emph{ae} [aj] and \emph{au} [aw]---are found in no more than a handful of words.
The diphthong \emph{eu} [eu] occurs in the interjections \emph{heu} `alas!' and \emph{heus} `hey!', the contractions \emph{ceu} `??' ($<$ \emph{ci-ve}) \emph{neu} `??' ($<$ \emph{ni-ve}), and \emph{seu} `??' ($<$ \emph{si-ve}), and a few Greek borrowings (e.g., \emph{eunuchus} `eunuch').
The diphthong \emph{oe} [oe] merged with \emph{u} in the pre-Classical period, but persists in a few words (e.g., \emph{poena} `punishment').
The word \emph{prout} `insofar', historically a contraction *\emph{pro-ut}, is the only source of \emph{ou} [ou].
Finally, $\langle$VI$\rangle$ only in 
\emph{uj} 

\subsection{Glides}
\sindex[subject]{Glides}

\section{Vowel-glide alternations}
\sindex[subject]{Vowel-glide alternations!Vocalization}
\sindex[subject]{Vowel-glide alternations!Gliding}

\citet{Hill1954}
\footnote{The historical context for this study may not be obvious to modern-day readers; \citeauthor{Hill1954} presents evidence for}

\citet{Steriade1984}

\begin{example}[Surface glides]
\vspace{\baselineskip}
\begin{tabular}{l c ll c ll c ll c ll}
%a. & \emph{jocus}   & `jest'       && \emph{vōtum}  & `prayer, vow'         \\
%   & \emph{jecur}   & `liver'      && \emph{vīnea}  & `vineyard'            \\
%%  & \emph{jactāre} & `to throw'   && \emph{vacāre} & `to be empty'         \\
%%  & \emph{jubēre}  & `to command' && \emph{virēre} & `to sprout, flourish' \\
%  & \emph{jubēre}  & `to command' && \emph{vacāre} & `to be empty'         \\
a. && \emph{jocus}    & `jest'      && \emph{jecur}  & `liver'     && \emph{jubēre} & `to command' \\
   && \emph{vōtum}    & `vow'       && \emph{vīnea}  & `vineyard'  && \emph{vacēre} & `to be empty' \\
b. && \emph{plēbējus} & 'plebeian'  && \emph{Mājus}  & `May'       && \emph{mejere} & `to urinate' \\
   && \emph{naevus}   & `birthmark' && \emph{clāva}  & `cudgel'    && \emph{novāre} & `to renew' \\
\end{tabular}
\end{example}

Two details bear mentioning. First, 

\emph{cuius} `whose'
[kuj.jus]
\emph{ui}
\emph{pējor} `worse' 
[peːj.jor]
\emph{ei}

other diphthongs:

\emph{ae}
\emph{oe}
\emph{ei}

\emph{au}
\emph{eu}

\citet{Devine1977}

\section{Height alternations}
\sindex[subject]{Height alternations!Lowering}

\section{Epenthesis}
\sindex[subject]{Epenthesis}

\section{Truncation}
\sindex[subject]{Truncation}
