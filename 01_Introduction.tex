\chapter{Introduction}
\label{introduction}

\section{Defining Classical Latin}

This monograph attempts to provide a comprehensive analysis of the synchronic phonology of Classical Latin.
The designation ``Classical'' here refers to the literary language of the Roman Empire from the earliest reasonably well-attested author (playwright Titus Maccius Plautus, born 254 BCE) to the death of emperor Marcus Aurelius Antoninus Augustus in 180 CE.
%, stoic philosopher-king and last of the Five Good Emperors, in 180 CE.

Over this long span of time---four centuries in which Rome evolved from a small republic to a sprawling empire---and space---the frontiers of the empire at times extended as far north as Scotland and as far east as the Persian Gulf---the literary language is remarkably stable.
Placing the earliest authors under the Classical umbrella is a slight departure from philological tradition, but in the eyes a the linguist the object of inquiry here is considerably less artificial than modern-day labels like ``English'' or ``Arabic''.

In part, grammatical uniformity over such a span of time and space is the product of leveling resulting from centuries of textual transmission;
for some texts, the oldest extant manuscript is separated from the original by more than a millenium.
Perhaps as a consequence, little of what variation remains can be related either to location or to date of authorship.
Another factor is that nearly all extant texts were written by members of the same milieu.
Education, and thus literacy, was the prerogative of the Roman elite---there were few if any public schools---and largely the prerogative of males, though some upper class families retained tutors for their daughters.
That some women achieved a high degree of literacy can be inferred from the satires of Juvenal (Decimus Junius Juvenalis), who writes of his loathing for women of letters, and protests that \emph{soloecismum liceat fēcisse marītō} `a husband ought to be allowed a solecism' (\emph{Satires} 6.455).
Unfortunately, very little written by Classical women has survived.\footnote{
    Ironically, one of these fragments, consisting of two lines of poetry, is preserved by the scholia to Juvenal.
    Some (male) scholars have challenged the attribution of this poem to a woman on the grounds that it is either too sensual---the reading of the first verse is disputed, but the second reads \emph{nudam Caleno concubantem proferat} `naked, making love to Calenus'---or too sophisticated for a Roman woman \citep[see][]{Keith2006}.}
But despite the narrow background of most Roman authors, there is evidence that other registers of Latin were spoken alongside the written standard.
These registers can be inferred from graffiti, grammatical prescriptions, and dialogue from plays, especially those of Terence (Publius Terentius Afer), himself a freed slave of North African descent.
None of this should imply that the written standard was artificial. 
It is only later, in the 3rd century CE and beyond, that it becomes apparent that the the written standard no longer corresponds to a spoken register.

\subsection{Reconstructing Classical Latin pronunciation}

Classical Latin orthography is ``shallow'', allowing the careful observer to infer much about Classical pronunciation.
Other evidence comes from graffiti, poetry, the testimony of contemporary grammarians, and adaptation and transcription (i.e., romanization) of loanwords.
In interpreting this evidence, the analyses of \citealt{Allen1978} have been particularly valuable.

With a few exceptions, all clearly indicated, the Latin orthography has been used here for transcription, with two minor following modifications.
First, the Latin orthography does not distinguish between high vowels [i, u] and glides [j, w], respectively; for example, the word [juwenis] `young' is spelt $\langle$IVVENIS$\rangle$.
Since the conditions governing the appearance of glides are of considerable importance, the front and back glides are indicated with \emph{j} and \emph{v}, respectively; hence \emph{juvenis}.
%despite the fact that neither character were actually known to the Romans.
Second, though short and long monophthongs are contrastive, they are rarely distinguished in Classical texts (though see \citealp{Rolfe1922}).
Macrons are therefore used to distinguish the long monophthongs \emph{ā}, \emph{ē}, \emph{ī}, \emph{ō}, \emph{ū} from the short monophthongs 
\emph{a}, \emph{e}, \emph{i}, \emph{o}, \emph{u}.
Since macrons do not usually appear in modern editions either, they have been restored using the Oxford Latin Dictionary \citep{OLD}.
These in turn can be confirmed by diachronic developments---short and long monophthongs usually have different reflexes in the Romance languages---or by Latin poetry---quantitative metres distinguish short and long vowels in some contexts.
%we consult an etymological dictionary of Romance \citep{Meyer-Lubke1935}, 

(\ref{alphabet}) gives the Roman character inventory, the orthography used in this monograph, and a phonetic transcription in the International Phonetic Alphabet.
Further details are given in the following three chapters.

%\begin{table}
\begin{example}[The Latin alphabet]
\begin{minipage}{\linewidth}
\centering
\begin{tabular}{C{10mm}@{ }C{10mm}@{ }C{10mm}@{ }C{10mm}@{ }C{10mm}@{ }C{10mm}@{ }C{10mm}@{ }C{10mm}@{ }C{15.5mm}@{ }C{10mm}@{ }C{10mm}@{ }C{5.5mm}}
A         & B        & C        & D        & E        & F        & G\footnote[1]{Indistinguishable from $\langle$C$\rangle$ in some early texts} & H        & I                  & K\footnote[2]{Largely restricted to Greek loanwords\label{marge}} & L        \\
\emph{a}, \emph{ā}  & \emph{b} & \emph{c} & \emph{d} & \emph{e}, \emph{ē} & \emph{f} & \emph{g} & \emph{h} & \emph{i}, \emph{ī}, \emph{j} & \emph{k} & \emph{l} \\
\pad{}[a, aː] & [b]      & [k]      & [d]      & [e, eː]      & [f]      & [g, ŋ]   & [h]      & [i, iː, j]             & [k]      & [l]      \\
\\
M         & N        & O        & P        & Q\footnote[3]{Only used before \emph{u} (\emph{qu} = [kw])} & R        & S        & T        & V                  & X        & Y\textsuperscript{\ref{marge}}        & Z\textsuperscript{\ref{marge}}        \\
\emph{m}  & \emph{n} & \emph{o}, \emph{ō} & \emph{p} & \emph{q} & \emph{r} & \emph{s} & \emph{t} & \emph{u}, \emph{ū}, \emph{v} & \emph{x} & \emph{y} & \emph{z} \\
\pad{}[m] & [n, ŋ]   & [o, oː]      & [p]      & [k]      & [r]      & [s]      & [t]      & [u, uː, w]             & [ks]     & [y]      & [z]      \\
\end{tabular}
\end{minipage}
%\caption{The Latin alphabet in majiscule, miniscule, and phonetic transcription.}
\label{alphabet}
\end{example}

The extant manuscripts from the Classical era (so defined) cover a wide variety of genres including oratories, histories, legal documents, manuals for agriculture and medicine, grammars, comedies and dramas, novels, and poems frivilous, erotic, or epic.
Some of these texts were only preserved thanks to bizarre coincidence.
For example, a 4th century manuscript, the Ambrosian palimpsest, is the only source for fragments of Plautus.
Sometime in the middle ages, the original text was scrubbed away and overwritten with liturgical texts of no value, but despite this the original remains legible in places.

Unless otherwise noted, wordforms cited here are taken from the \emph{Bibliotheca Teubneriana Latina}, an electronic corpus consisting of critical editions of all extant texts from Plautus to the death of Marcus Aurelius.
The study of lexical and phrasal prosody in chapter \ref{wordphraseprosody} specifically draws upon the first six books of the \emph{Aeneid}, an epic poem written by Vergil (Publius Vergilius Maro) between 29--19 BCE; the edition used is that of \citet{Pharr1964}, who supplies macrons throughout.
Forms not found in this corpus, whether reconstructed, inferred from incomplete paradigms, or labeled as ungrammatical, are indicated with a star (*).

%Like most Latin authors, the work of Plautus is primarily attested in High Medieval manuscripts, though some of the surviving Plautus are preserved only in a 4th century manuscript, the Ambrosian palimpsest.
%Several centuries after it was written, the ink of the original pagan manuscript was scrubbed off and overwritten with liturgical texts of little distinction---though the original text remains legible in places.
%Recent innovations in textual preservation, including infrared, ultraviolet, and X-ray imaging, may soon allow for more to be divined from such marvelous accidents of history.
% attempts to develop a contemporary myth connecting the ruling families of Rome to the mythical Trojan hero Aenēās, and to glorify \emph{pietās} `sense of duty'.
%The book appears to have been published almost immediately after Vergil's death with minimal editorial intervention---in fact, there are dozens of incomplete lines.
%, in violation of the author's dying wish that the nearly-complete work be destroyed.
%\emph{Pseudolus}, a comic play written by Plautus around 200 BCE.
%\footnote{The spelling \emph{Virgil} appearing during the Renaissance may be a conscious attempt to evoke \emph{virgō} `maiden': the author, a ``virtuous pagan'', died a bachelor.}

\subsection{Diachronic explanation}

The details of Classical Latin's painstakingly reconstructed prehistory have been carefully documented in numerous authoritative sources,
among them an excellent handbook by \citet{Sommer1902} and a comparative Greek-Latin grammar by \citet{Sihler1995}.
Consequently, the prehistory of Latin is given very little ink here, except to draw analogies between synchronic and diachronic patterns.
It has been clear at least since Saussure that diachronic and synchronic facts and explanations are essential incomparable; the former consists of a relation between the surface forms of ``earlier'' and ``later'' grammars, whereas the latter consists of input-output relationship within a single grammar.
Only the latter is within the scope of the current work.

\section{Theoretical assumptions}

\citet{SPE}
\citet{Halle1993}

A useful summary of the different senses is provided by \citet[chap.~1]{Blaho2008}.
\subsection{Rule application}

is that the
\textsc{Level Ordering} \citep{Siegel1974} is inconsistent with the relevant data. This fact which has been recognized for decades \citep{Aronoff1976} but has unfortunately been tragically ignored.

\subsection{Prosodic representations}

\section{Substance in phonology}

To deny that one can study the phonology in this way would be equivalent to denying that one can study a computer program independent of

The underlying segment indicated /k/, for instance, need not have any auditory or articulatory substance. 
It is simply an abbreviation for the mental entity which, barring further complications is realized as a sequence of complete velar closure followed by a pulmonic gesture and the onset of larnygeal vibration; in other words, [k].
While it is not implausible that such mental entity /k/ has additional ``substance'', there is of yet no compelling evidence that any additional details of articulatory or acoustic realization are accessible during grammatical computation.

%I wish to go one step farther than most phonologists in assuming
%that that features and feature values are equally abstract in nature, and linked to no acoustic/auditory or articulatory substance.
%Despite this, I adopt traditional feature labels: /k/ is, for instance, [$-$\textsc{Voice}].
%However, my suspicion is that [$-$\textsc{Voice}] is just an abbreviation for the set of segments which devoice a preceding obstruent (see \S?); the fact that they themselves were realized with a less periodic laryngeal vibration is, from the point of view of the synchronic grammar, irrelevant.

While perhaps most linguists would expect something called a grammar to provide an analysis which generate correct surface forms given certain underlying representations, this is unfortunately not always the case (\citealp[19]{Bauer2001}, \citealp[23]{Hall1997}).

\subsection{Principles of morphology}

\citet{Matthews1972a} is an enthusiastic proponent.
More tempered positions are taken by \citet{Lieber1980} and \citet{Aronoff1994}.

\section{Outline}


chicken-and-egg problem of simultaneously 
presenting relevant surface forms, explicating numerous interacting rules, and justifying the underlying representations to which these rules apply.
Chapter \ref{tutorial} provides a tutorial overview of Latin phonology, illustrating the phonological processes while largely neglecting formal details.
Chapters \ref{consonants}--\ref{vowels} present the inventories of consonant, vowel, and glide inventories and major segmental rules.











For instance, \citet{Robins1979} writes that, ``with the rest of western antiquity, Priscian's grammatical model is word and paradigm, and he expressly denies any linguistic significance to divisions\ldots{}below the word.'' (56)
Word-level prosody, including word stress, is the subject of chapter \ref{wordpros}.
