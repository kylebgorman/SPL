\chapter{Introduction}

\section{Classic Latin in space and time}

This book attempts to provide a relatively comprehensive account of the synchronic phonology of Classical Latin, by which is meant the literary language of the Roman Empire from the earliest reasonably well-attested author (playwright Titus Maccius Plautus, born 254 BCE) to the death of emperor Marcus Aurelius (180 CE), stoic philosopher-king and last of the Five Good Emperors.

\subsection{Manuscript tradition}

In these four centuries, Rome grew from a small republic to a sprawling empire.
Despite the span of time and enormous size of the empire---at times the frontiers extended as far north as Scotland and as far east as the Persian Gulf---the language is remarkably stable across time and space.
Centuries of textual transmission has surely leveled some variability present in earlier manuscripts, and little of what remains can be related either to location or date of authorship.

One reason for this may be that education, and thus literacy, was the perogative of the elite; there were few if any public schools.
%One exception is the playwright Terence, a freed slave.
Literacy was also largely the prerogative of males, though some upper class families retained tutors for their daughters.
The satirist Juvenal (Decimus Junius Juvenalis), writes of his loathing for women who love literature and protests that \emph{soloecismum liceat fēcisse marītō} `a husband ought to be allowed a solecism' (\emph{Satires} 6.455).
Despite the evidence that many women were literate, all that has survived of women's writing are a few letter fragments and two lines of erotic poetry have survived.\footnote{
    Ironically, the latter is preserved only in the scholia to Juvenal.
    Some (male) scholars have challenged the attribution of this poem to a woman on the grounds that it is either too sensual---the first verse is obscure and many emendations have been proposed, but the second reads \emph{nudam Caleno concubantem proferat} ``naked, making love to Calenus''---or too sophisticated for a Roman woman \citep[see][]{Keith2006}.}

But despite the narrow background of most Roman authors, dialogue from fictional works (especially those of Terence, himself a freed slave) and grammatical prescriptions indicate that lower registers of Latin were spoken alongside the Classical language.
Only later is it clear that spoken and written Latin have truly diverged.

Extant manuscripts cover a wide variety of genres including oratories, histories, legal documents, manuals for agriculture and medicine, grammars, comedies and dramas, novels, and poems frivilous, erotic, or epic.

Often these texts have been preserved thanks only to a confluence of bizarre circumstances.
Like most Latin authors, the work of Plautus is primarily attested in High Medieval manuscripts.
But some of the surviving Plautus are preserved only in a 4th century manuscript known as the Ambrosian palimpsest.
Several centuries after it was written, this pagan manuscript was ``recycled''---the ink was scrubbed off and overwritten with liturgical texts---though the original text remains legible in places.
Recent innovations in textual preservation---such as the use of infrared, ultraviolet, and x-ray imaging---may soon allow for more to be divined from such marvelous accidents of history.

\subsection{Reconstructing pronunciation}

Latin has a shallow orthography, one which allows the careful observer to infer much about Classical pronunciation.
Other evidence comes from graffiti, the testimony of the grammarians, and adaptation (or simply transcription, i.e., romanization) of loanwords.
The interpretation used here draws extensively---though not uncritically---upon \citealt{Allen1978}.
Romance reflexes provide further evidence, and are taken from \citealt{Meyer-Lubke1935}, an etymological dictionary of Romance.

With a few exceptions, all clearly indicated, the Latin orthography is also used here for transcription purposes. 
There is but one serious limitation of this orthography, which is that it does not distinguish between high vowels [i, u] and glides [j, w] respectively; 
for instance, the word [juwenis] `young' is spelt $\langle$IVVENIS$\rangle$.
Since the conditions governing the appearance of glides will be of considerable importance in late chapters, here glides are indicated with \emph{j} and \emph{w}, despite the fact that neither character were actually known to the Romans.

\begin{example}[The Latin alphabet]
\centering
\vspace{\baselineskip}
\begin{tabular}{cccccccccccc}
A         & B        & C        & D        & E        & F        & G\footnote[1]{
    Indistinguishable from $\langle$C$\rangle$ in some early texts}             & H        & I                  & K\footnote[2]{ 
    Largely restricted to Greek loanwords\label{marge}}& L & M \\
\emph{a}  & \emph{b} & \emph{c} & \emph{d} & \emph{e} & \emph{f} & \emph{g} & \emph{h} & \emph{i}, \emph{j} & \emph{k} & \emph{l} & \emph{m} \\
\pad{}[a] & [b]      & [k]      & [d]      & [e]      & [f]      & [g, ŋ]   & [h]      & [i, j]             & [k]      & [l] & [m] \\
\end{tabular}

\vspace{\baselineskip}

\begin{tabular}{cccccccccccc}
N            & O        & P        & Q\footnote[3]{
    Only used before \emph{u}}                                       & R        & S        & T        & V                  & X        & Y\textsuperscript{\ref{marge}}                               & Z\textsuperscript{\ref{marge}} \\
\emph{n}     & \emph{o} & \emph{p} & \emph{q}                        & \emph{r} & \emph{s} & \emph{t} & \emph{u}, \emph{v} & \emph{x} & \emph{y} & \emph{z} \\
\pad{}[n, ŋ] & [o]      & [p]      & [k]                             & [r]      & [s]      & [t]      & [u, w]             & [ks]     & [y]      & [z] \\
\end{tabular}
\end{example}

%I have chosen to spend more time interacting with the phonological literature that incidentally discusses Latin phenomena, rather than with grammars or handbooks of Latin.
%However, the handbook of \citet{Sommer1902} has served as a useful source of generalizations both synchronic and diachronic.

\subsection{Diachronic explanation}

Handbooks extensive details about Classical Latin's painstakingly reconstructed prehistory.
In contrast, these details are given little ink here except to direct interested readers to the authoritative sources, or to draw an illuminating analogy between synchronic and diachronic patterns.



It has been clear at least since Saussure that diachronic and synchronic relationships have distinct inputs and cannot really be compared; the former consists of a relation between the outputs of ``earlier'' and ``later'' grammars, whereas the latter consists of input-output relationship within a single grammar.

This is due to my own conviction that producing a ``correct'' description of Latin is only of scientific interest insofar as it reveals the possibilities of the human language facility.
Whatever Plautus must have known as native speaker of Classical Latin must to a first approximately, also have been known by any other typically-developing infant with no hearing impairments born under similar circumstances.
This is not to say that all who call themselves linguists should focus solely on X, but that this is a necessary if the products of linguistic analysis can be said to be more or less ``correct'' rather than more or less aesthetically pleasing.

and a grammar by \citet{Sommer1902}
the comparative Latin grammar by \citet{Sihler1995}

\emph{Pseudolus}, a comic play written by Plautus around 200 BCE.

The \emph{Aeneid}, an epic poem written by Vergil (Publius Vergilius Maro)\footnote{
    The spelling \emph{Virgil} appearing during the Renaissance may be a conscious attempt to evoke \emph{virgō} `maiden': the author, a ``virtuous pagan'', died a bachelor.}
between 29 BCE and his death 10 years later, attempts to develop a contemporary myth connecting the ruling families of Rome to the mythical Trojan hero Aenēās, and to glorify \emph{pietās} `sense of duty'.
The book is thought to have been published almost immediately after Vergil's death, with minimal editorial intervention---in fact, there are dozens of incomplete lines.
%, in violation of the author's dying wish that the nearly-complete work be destroyed.

\section{Theoretical assumptions}

\citet{SPE}
\citet{Halle1993}

A useful summary of the different senses is provided by \citet[chap.~1]{Blaho2008}.
\subsection{Rule application}

is that the
\textsc{Level Ordering} \citep{Siegel1974} is inconsistent with the relevant data. This fact which has been recognized for decades \citep{Aronoff1976} but has unfortunately been tragically ignored.

\subsection{Prosodic representations}



\section{Substance in phonology}

To deny that one can study the phonology in this way would be equivalent to denying that one can study a computer program independent of

The underlying segment indicated /k/, for instance, need not have any auditory or articulatory substance. 
It is simply an abbreviation for the mental entity which, barring further complications is realized as a sequence of complete velar closure followed by a pulmonic gesture and the onset of larnygeal vibration; in other words, [k].
While it is not implausible that such mental entity /k/ has additional ``substance'', there is of yet no compelling evidence that any additional details of articulatory or acoustic realization are accessible during grammatical computation.

%I wish to go one step farther than most phonologists in assuming
%that that features and feature values are equally abstract in nature, and linked to no acoustic/auditory or articulatory substance.
%Despite this, I adopt traditional feature labels: /k/ is, for instance, [$-$\textsc{Voice}].
%However, my suspicion is that [$-$\textsc{Voice}] is just an abbreviation for the set of segments which devoice a preceding obstruent (see \S?); the fact that they themselves were realized with a less periodic laryngeal vibration is, from the point of view of the synchronic grammar, irrelevant.

many grammars do contain rules for generating ungrammatical phonological strings \citep[19]{Bauer2001}.
% look up Botha
