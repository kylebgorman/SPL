\chapter{Introduction}

This book attempts to provide a relatively comprehensive account of the synchronic phonology of Classical Latin, by which we mean the literary language of the Roman Empire from the earliest reasonably well-attested author (the dramatist, Plautus, born 254 BCE) to the time of Constantine the Great and the Christianization of the Empire.
%as it was used by Plautus, Cicero, Vergil, Seneca, Martial, Ovid, Tacitus, and other authors
The manuscripts handed down to us include scripts for comedies and dramas, public oratories, poems both frivolous and epic, histories, legal documents, agricultural manuals, and even descriptive grammars.
The language of this period was written in a shallow orthography which allows the careful observer to deduce much about the pronunciation of a ``high" form of Latin during this era, and evidence from contemporary grammatical descriptions, graffiti, and reflexes in Romance fill in many of the details.
For interpreting Latin orthography, I have relied primarily upon \citet{Allen1978}
%I have chosen to spend more time interacting with the phonological literature that incidentally discusses Latin phenomena, rather than with grammars or handbooks of Latin.
%However, the handbook of \citet{Sommer1902} has served as a useful source of generalizations both synchronic and diachronic.

This is due to my own conviction that producing a ``correct'' description of Latin is only of scientific interest insofar as it reveals the possibilities of the human language facility.
Whatever Plautus must have known as native speaker of Classical Latin must to a first approximately, also have been known by any other typically-developing infant with no hearing impairments born under similar circumstances.
This is not to say that all who call themselves linguists should focus solely on X, but that this is a necessary if the products of linguistic analysis can be said to be more or less ``correct'' rather than more or less aesthetically pleasing.


\section{Data sources}


\subsection{Wordlist}



\subsection{Scanned poetry}

\emph{Pseudolus}, a comic play written by Titus Maccius Plautus around 200 BCE.

The \emph{Aeneid}, an epic poem written by Publius Vergilius Maro (Vergil),\footnote{
    The spelling \emph{Virgil} that appears in the Renaissance may be intended to evoke \emph{virgō} `maiden'. Vergil was the Proust of his era---effete, sickly, sheltered, and a lifelong bachelor with few cares beyond his \emph{magnum opus}.}
between 29 BCE and his death 10 years later, attempts to develop a contemporary myth connecting the ruling families of Rome to the mythical Trojan hero Aenēās, and to glorify \emph{pietās}.\footnote{
    ``An attitude of dutiful respect towards those to whom one is bound by ties of religion, consanguinity, etc.'' \OLD{123}}
The book is thought to have been published almost immediately after Vergil's death, with minimal editorial intervention---in fact, there are dozens of incomplete lines.
%, in violation of the author's dying wish that the nearly-complete work be destroyed.

\section{Transcription}

\section{Theoretical assumptions}

\subsection{Prosodic representations}

\subsection{Rule application, ordering, opacity}

is that the
\textsc{Level Ordering} \citep{Siegel1974} is inconsistent with the relevant data. This fact which has been recognized for decades \citep{Aronoff1976} but has unfortunately been tragically ignored.

\section{Substance in phonology}

To deny that one can study the phonology in this way would be equivalent to denying that one can study a computer program independent of

I do not assume, therefore, that the underlying segment indicated /k/ is linked to any auditory or articulatory substance; whatever the internals of /k/, it is simply an abbreviation for the mental entity which, barring further complications, which is realized as a complete voiceless velar closure/release superimposed on a pulmonic XXX; as it is conventionally known, [k].
While it is not implausible that such mental entity /k/ has additional ``substance'', there is of yet no compelling evidence that any additional details of articulatory or acoustic realization are accessible during grammatical computation.

%I wish to go one step farther than most phonologists in assuming
%that that features and feature values are equally abstract in nature, and linked to no acoustic/auditory or articulatory substance.
%Despite this, I adopt traditional feature labels: /k/ is, for instance, [$-$\textsc{Voice}].
%However, my suspicion is that [$-$\textsc{Voice}] is just an abbreviation for the set of segments which devoice a preceding obstruent (see \S?); the fact that they themselves were realized with a less periodic laryngeal vibration is, from the point of view of the synchronic grammar, irrelevant.

\section{The role of diachrony}

In many cases, there are established pre-historical ``causes'' for the details of a phonological pattern in Classical Latin.
In general, I have chosen not to refer to these diachronic facts except to providel references to the appropriate handbooks, or to draw an illuminating analogy between the diachronic and synchronic patterns.
It has been clear at least since Saussure that diachronic and synchronic relationships have distinct inputs and cannot really be compared; the former consists of a relation between the outputs of ``earlier'' and ``later'' grammars, whereas the latter consists of input-output relationship within a single grammar.
