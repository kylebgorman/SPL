\chapter*{Preface}

My intentions for this monograph are threefold.

Beyond the narrow scope of Latin, this study investigates the structure of geminate consonants, long monophthongs, and diphthongs, and specifically the interaction between geminate consonant and vowels, something which has received surprisingly little attention.
Furthermore, this grammar provides some clues into the nature of non-derived environment blocking and lexical exceptionality, which, in the author's opinion, can only be studied in the context of a relatively exhaustive phonological grammar.
Classical Latin, as a well-documented dead language, provides the ideal tool for investigating these questions: a large, carefully edited closed corpus.

This monograph is structured in a somewhat different fashion than is typical for phonological grammars.
After introducing the material and theoretical assumptions, chapter \S\ref{?} provides a ``tutorial'' in Latin phonology, introducing all but few phonological processes and summarizing the phonological analysis.
The following 5 chapters are dedicated to clarifying and motivating the analysis in greater detail.
The final 3 chapters apply the analysis to show that the analysis generates the inflectional and derivational paradigms of Classical Latin.

\noindent Kyle Gorman \\ Portland, OR
