\chapter*{Preface}

My intentions for this monograph are threefold.

assimilation
assibilation
vowel length
vowel/glide interactions

Beyond the narrow scope of Latin, this study investigates the structure of geminate consonants, long monophthongs, and diphthongs, and specifically the interaction between geminate consonant and vowels, something which has received surprisingly little attention.

Furthermore, this grammar provides some clues into the nature of non-derived environment blocking, rule ordering, and lexical exceptionality, which can only be studied in the context of a relatively exhaustive phonological grammar.

Classical Latin, as a well-documented dead language, provides the ideal tool for investigating these questions: a large, carefully edited closed corpus.

This monograph is structured in a somewhat different fashion than is typical for phonological grammars.
After introducing the material and theoretical assumptions,
The following 5 chapters are dedicated to clarifying and motivating the analysis in greater detail.
The final 3 chapters apply the analysis to show that the analysis generates the inflectional and derivational paradigms of Classical Latin.

Comments from X, Y, ZX, Y, ZX, Y, ZX, Y, Z
reviewers and editors at the journals \emph{Language} and \emph{Phonology}

While the intellictual debts of this monograph extend back millennia, I would like to highlight the important contributions made by Stephen Anderson's work into the history of generative phonology, Charles Yang's operationalization of morphophonological productivity, Richard Sproat's formalization of the Latin verb. But most of all the mentalism of Roman Jakobson, Morris Halle, and Noam Chomsky, \emph{sine qua nōn}.

\hfill\textsc{K.B.G.}

\noindent
\textsc{Portland, OR}\\
\textsc{\monthname~\the\year}
