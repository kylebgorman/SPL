\chapter*{Preface}
\label{preface}

My intentions for this monograph are threefold.

the the 

thingo there there

assimilation
assibilation
vowel length
vowel/glide interactions

This study also considers Latin segmental phonology 

In particular, observations regarding the interactions between geminate consonants and long vowels, and the relationship between glides and high vowels, may serve to illuminate the structural representation of such segments in all natural languages.

Finally, this grammar provides some clues into the nature of non-derived environment blocking, rule ordering, and lexical exceptionality, which can only be studied in the context of a relatively exhaustive phonological grammar.

Classical Latin, as a well-documented dead language, provides the ideal tool for investigating these questions: a large, carefully edited closed corpus, a corpus which is remarkably uniform given the large expanses of space and time separating Plautus from 

This monograph is structured in a somewhat different fashion than is typical for phonological grammars.
After introducing the material and theoretical assumptions,
The following 5 chapters are dedicated to clarifying and motivating the analysis in greater detail.
The final 3 chapters apply the analysis to show that the analysis generates the inflectional and derivational paradigms of Classical Latin.

Comments from X, Y, ZX, Y, ZX, Y, ZX, Y, Z
reviewers and editors at the journals \emph{Language} and \emph{Phonology}

%While the intellictual debts of this monograph are deep, I would like to highlight the important contributions made by Stephen Anderson's study of the history of generative phonology and Charles Yang's work on morphophonological productivity, and studies of the Latin verb by David Embic and Richard Sproat.
%But most of all the mentalism of Roman Jakobson, Morris Halle, and Noam Chomsky, without which 
%\emph{sine qua nōn}.

\hfill\textsc{KBG}
\vspace{\baselineskip}

\noindent
\textsc{Portland, OR}\\
\textsc{\monthname~\the\year}
